%%%---> page 1
\AddToTemplate{LugaCover}
\AddToTemplate{Navigatorbarbg}
\AddToTemplate{LugatexLogo}

\vglue 130pt
\hspace{-5pt}
\mbox{%
\colorlet{myorange}{red!30!yellow}
\textcolor{cyan}{{\tiny \framebox{Просмотр документа возможет только
программой:}\kern0bp\colorbox{cyan}{\textcolor{white}{Adobe Reader версии 8.0 и выше}}}}
\pdfcomment[subject={Аннотация},author={ВНИМАНИЕ! прочитайте перед запуском
программы},color={1 0 0},
voffset=4pt,opacity=0.5,open=true]{Для просмотра демонстрационной версии LugaTeX,
требуется установить на вашей операционной системе ($\text{Windows}^*$,
Linux, Solaris, $\text{Unix}^*$) программу Adobe Reader ВЕРСИИ 9.0.
и выше. Все функции в системе могут быть активированы и использованы только
с помощью программы Adobe Reader. Другие программы для отображения формата
PDF не подерживаются}}

\vfill
\mbox{}
\hfill
\mbox{%
\textcolor{cyan}{\underline{{\tiny Загрузить программу Adobe Reader 9.0, можно
$\text{здесь}^\text{%
\mbox{\href{ftp://10.2.61.3/Adobe-Reader/}{\mbox{\includegraphics[scale=.03]{Adobe_Reader_01.png}}}}}$
или на сайте компании $\text{Adobe}^\text{\mbox{%
\href{http://get.Adobe.com/reader}{\mbox{\includegraphics[scale=.03]{Adobe_Reader_02.png}}}}}$}}}}

\newpage
\disableTemplate{LugaCover}
\AddToTemplate{LugaNotes}
\AddToTemplate{Navigatorbarbg}
\AddToTemplate{LugatexLogo}

\begin{flushleft}
\begin{tikzpicture}
\tikzstyle{rec} = [rectangle,rounded corners,ultra thick,draw=monred!45] 
\node[rec,text=white,inner sep=3mm,drop shadow] (th){\parbox{0.6\textwidth}{%
\parbox{245pt}{%
\scriptsize{Интерактивная система знаний представляет собой интерактивные формы
предоставления информации. Система разработана для создания интерактивных
учебных лекцый по разным дисциплинам университетских программ, и
экзаменационно --- модульных заданий ввиде интеративных форм заполнения и
контроля заданий. Ответы могут быть представлены в различной форме или
зашифрованы специальной формой. Вся система может быть зашифрована с 
использованием пароля, печать, выделение текста, сохранение или скриншоты
системы знаний могут быть заблокированы преподователем.
Экзаменационные и модульные контроли могут быть представлены в различных
формах, с возможностью отображения правельного решения или просто ответа
или ввиде ссумы очков правельных ответов. Система позволяет включать
как аудио так и видио материалы паралельно с лекциями. Различные
интерактивные формы в учебных материаллах демонстрируют разнообразие
и простоту их использования демонстрируя наглядность информации с
различными формами предоставления знаний.}}
}};
\node[rec,text=myborder,anchor=south west,xshift=3mm,yshift=1mm]
at (th.north west) {\small Интерактивная система знаний };
\end{tikzpicture}
\end{flushleft}

\vglue 40pt

\hspace{-32pt}
\parbox{350pt}{%
\pushButton[\ui{%
	bordercolor={0.8 0.85 1},bgcolor={0.9 0.85 1},
	textcolor={0 0 0},align={right},
	uptxt={Аудио-Видео},
	js={this.pageNum=3}
}]{pb1}{60bp}{11bp}\kern1bp\pushButton[\ui{%
	bordercolor={0.8 0.85 1},bgcolor={0.9 0.85 1},
	textcolor={0 0 0},align={right},
	uptxt={Диаграммы},
	js={this.pageNum=25}
}]{pb1}{61bp}{11bp}\kern1bp\pushButton[\ui{%
	bordercolor={0.8 0.85 1},bgcolor={0.9 0.85 1},
	textcolor={0 0 0},align={right},
	uptxt={Анимация},
	js={this.pageNum=26}
}]{pb1}{61bp}{11bp}\kern1bp\pushButton[\ui{%
	bordercolor={0.8 0.85 1},bgcolor={0.9 0.85 1},
	textcolor={0 0 0},align={right},
	uptxt={3D Графика},
	js={this.pageNum=54}
}]{pb1}{61bp}{11bp}\kern1bp\pushButton[\ui{%
	bordercolor={0.8 0.85 1},bgcolor={0.9 0.85 1},
	textcolor={0 0 0},align={right},
	uptxt={Элект Формы},
	js={this.pageNum=53}
}]{pb1}{61bp}{11bp}\kern1bp\pushButton[\ui{%
	bordercolor={0.8 0.85 1},bgcolor={0.9 0.85 1},
	textcolor={0 0 0},align={right},
	uptxt={Автор},
	js={app.alert("\u0410\u0434\u0432\u043e\u043a\u0430\u0442\u0443\u0440\u0430")}
}]{pb1}{60bp}{11bp}\kern1bp\pushButton[\ui{%
	bordercolor={0.8 0.85 1},bgcolor={0.9 0.85 1},
	textcolor={0 0 0},align={right},
	uptxt={Заказ},
	js={this.pageNum=56}
}]{pb1}{60bp}{11bp}
}

\hfill
\begin{flushleft}
	\hspace{-20pt}
	\begin{tabular}{llll}
		\textcolor{myborder}{Теоретическая
		физика~\pdfmarkupcomment[color=blue,opacity=1.0,
		subject={Аннотация},author={Прочитайте!}]{}{%
		Пример лекции по теоретической физике, нажмите кнопку
	для перехода на страницу с лекциями, кнопка с названием задачи
	- для перехода на страницу с демонстрационными примерами решения
	задач по курсу, кнопка с названием модули - для перехода
	на страницу с интерактивными формами, для прохождения тестовых
	задач, итог тестов пердставлен в виде суммы правельных ответов }} &\pushButton[\ui{%
	bgcolor={0.9 0.85 1},
	textcolor={0 0 0},align={right},
	uptxt={Лекция},
	js={this.pageNum=3}
}]{pb1}{42bp}{11bp}\kern2bp\pushButton[\ui{%
	bgcolor={0.9 0.85 1},
	textcolor={0 0 0},align={right},
	uptxt={Задачи},
	js={this.pageNum=21}
}]{pb1}{42bp}{11bp}\kern2bp\pushButton[\ui{%
	bgcolor={0.9 0.85 1},
	textcolor={0 0 0},align={right},
	uptxt={Тесты},
	js={this.pageNum=24}
}]{pb1}{42bp}{11bp}	&	\mbox{}	&\\
	\textcolor{myborder}{Уравнения математической
		физики}~\pdfmarkupcomment[color=blue,opacity=1.0,
		subject={Аннотация},author={Прочитайте!}]{}{%
	Пример лекции по уравнениям математической физики, нажмите кнопку
	для перехода на страницу с лекциями, кнопка с названием задачи
	- для перехода на страницу с демонстрационными примерами решения
	задач по курсу, кнопка с названием модули - для перехода
	на страницу с интерактивными формами, для прохождения тестовых
	задач, итог тестов пердставлен в виде суммы правельных ответов }	&\pushButton[\ui{%
	bgcolor={0.9 0.85 1},
	textcolor={0 0 0},align={right},
	uptxt={Лекция},
	js={this.pageNum=28}
}]{pb1}{42bp}{11bp}\kern2bp\pushButton[\ui{%
	bgcolor={0.9 0.85 1},
	textcolor={0 0 0},align={right},
	uptxt={Задачи},
	js={this.pageNum=39}
}]{pb1}{42bp}{11bp}\kern2bp\pushButton[\ui{%
	bgcolor={0.9 0.85 1},
	textcolor={0 0 0},align={right},
	uptxt={Модули},
	js={this.pageNum=40}
}]{pb1}{42bp}{11bp} &   \mbox{} &\\
	\textcolor{myborder}{Функциональный
		анализ}~\pdfmarkupcomment[color=blue,opacity=1.0,
	subject={Аннотация},author={Прочитайте!}]{}{%
	Пример лекции по функциональному анализу, нажмите кнопку
	для перехода на страницу с лекциями, кнопка с названием задачи
	- для перехода на страницу с демонстрационными примерами решения
	задач по курсу, кнопка с названием модули - для перехода
	на страницу с интерактивными формами, для прохождения тестовых
	задач, итог тестов пердставлен в виде суммы правельных ответов }	&\pushButton[\ui{%
	bgcolor={0.9 0.85 1},
	textcolor={0 0 0},align={right},
	uptxt={Лекция},
	js={this.pageNum=41}
}]{pb1}{42bp}{11bp}\kern2bp\pushButton[\ui{%
	bgcolor={0.9 0.85 1},
	textcolor={0 0 0},align={right},
	uptxt={Задачи},
	js={this.pageNum=50}
}]{pb1}{42bp}{11bp}\kern2bp\pushButton[\ui{%
	bgcolor={0.9 0.85 1},
	textcolor={0 0 0},align={right},
	uptxt={Модули},
	js={this.pageNum=52}
}]{pb1}{42bp}{11bp}
\end{tabular}
\end{flushleft}
%%%--->	\newpage
%%%--->	
%%%--->	\begin{tikzpicture}[rounded corners,ultra thick]
%%%--->	\shade[top color=yellow,bottom color=black] (0,0) rectangle +(2,1);
%%%--->	\shade[left color=yellow,right color=black] (3,0) rectangle +(2,1);
%%%--->	\shadedraw[inner color=yellow,outer color=black,draw=yellow] (6,0)
%%%--->	rectangle +(2,1);
%%%--->	\shade[ball color=green] (9,.5) circle (.5cm);
%%%--->	\end{tikzpicture}
%%%--->	
%%%--->	\vglue 20pt
%%%--->	\hspace{250pt}
%%%--->	\scalebox{0.6}{%
%%%--->	\begin{minipage}{3cm}
%%%--->	\logo{green!80!black}{green!25!black}{green}{green!80}{leaf}\\
%%%--->	\logo{green!50!black}{black}{green!80!black}{red!80!green}{leaf}\\
%%%--->	\logo{red!75!black}{red!25!black}{red!75!black}{orange}{leaf}\\
%%%--->	\logo{black!50}{black}{black!50}{black!25}{}
%%%--->	\end{minipage}
%%%--->	}
