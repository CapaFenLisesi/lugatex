\newpage	%--- page 3
\disableTemplate{LugatexLogo}
\disableTemplate{Lugacontents}
\disableTemplate{Navigatorbarbg}
\AddToTemplate{Lugamechanicscover}
\AddToTemplate{LugatexLogo}
\AddToTemplate{Navigatorbarbg}

\vglue 60pt
\hspace{-12pt}
\parbox{180pt}{%
\begin{center}
	{\bf{\tiny Лекция основана на издании\\
\tooltip{Ландау~Л.~Д. Лифшиц~Е.~М.}{2}\\
\hyperlink{Landa}{Теоретическая физика}.\\
Том. I. --- \tooltip{Механика}{12}.}}
\end{center}}

\vglue 68pt
\hspace{32pt}
\mbox{\pushButton[\CA{Запуск}\RC{лекция}
\A{/S/Named/N/Quit}]{myButton5}{}{10bp}}
%\A{/S/Named/N/PrevPage}]{myButton5}{}{10bp}}

\vglue 35pt
\hspace{14pt}
%\begin{tikzpicture}
\iconbook~\tikz[every node/.style={signal,draw,text=white,signal to=nowhere}]
	\node[fill=red!65!black, signal to=east]  {{\bf {\tiny Полная версия}
	$\text{{\tiny издания}}^{\text{\pdfcomment[color=green,icon=Comment,
	subject={Аннотация},author={Как открыть скрытый документ}]{Нажмите на
	ярлык мышкой два раза, и лекции будут открыты в дополнительном
	окне программы Adobe Reader}}}$}};
%\end{tikzpicture}
~\attachfile[icon=Tag]{landay.pdf}
%%%---> ~\pdfcomment[color=blue,icon=Note,open=true]{This is another comment.}
%%%---> 
%%%---> \mbox{\Acrobatmenu{GeneralInfo}{here}}
%%%---> 
%%%---> \def\NavigationBar{{\Large 
%%%--->		\Acrobatmenu{FontsInfo}{\reflectbox{\ding{227}}}
%%%--->		\Acrobatmenu{NextPage}{\ding{227}}
%%%--->		\Acrobatmenu{FirstPage}{\reflectbox{\ding{224}}}
%%%--->		\Acrobatmenu{LastPage}{\ding{224}}
%%%--->		\Acrobatmenu{GoBack}{\reflectbox{\ding{249}}}
%%%--->		\Acrobatmenu{Quit}{\ding{54}}
%%%--->		\Acrobatmenu{GoBack}{\color{button}\Back}
%%%--->		\Acrobatmenu{GoForward}{\color{button}\Forw}
%%%--->		\Acrobatmenu{Close}{\color{button}\Close}
%%%--->		\Acrobatmenu{FullScreen}{\addButton{.85in}{Full Screen}}
%%%--->}}
