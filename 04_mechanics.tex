\newpage	%--- page 4
\disableTemplate{LugatexLogo}
\disableTemplate{Navigatorbarbg}
\disableTemplate{Lugamechanicscover}
\AddToTemplate{Lugamathbg}
\AddToTemplate{Navigatorbarbg}
\AddToTemplate{LugatexLogo}

\section{Классическая механика}
\subsection{Уравнения движения}
\subsubsection{Обобщенные координаты}
Основным понятием механики является понятие
\parbox{106pt}{%
\todo[inline, color=green!40]{\textbf{материальной точки}}}.\\


Под этим $\text{назва\-нием}^{\text{%
\pdfmarkupcomment[color=blue,opacity=1.0,subject={Аннотация},
author={Примечание}]{}{%
Вместо термина материлальная точка будем иметь ввиду о частицах}}}$
понимают тело, размерами которого можно пренебречь при
описании его движения.
Положение материальной точки в пространстве определяется ее
радиусом--вектором $r$, компоненты которого совпадают с ее декартовыми
координатами $x,\, y,\, z$. Производная $r$\, по времени $t$:
$$
\displaystyle
v = \frac{dr}{dt}
$$


называется скоростью, а вторая производная $ \frac{d^2 t}{dt^2}$\, ---
ускорением точки.


Для определения положения системы из $N$\, материальных точек в
пространстве надо задать $N$\, радиус--векторов, т.~е. $3N$\, координат.
Число независимых величин, задание которых необходимо для однозначного
определения положения системы, называется числом ее
\parbox{94pt}{\todo[inline, color=green!40]{\textbf{степеней свободы}}},
в нашем случае это число равно $3N$. Эти велечины не обязательно
должны быть декартовыми координатами точек, и в зависимости от условий
задачи может оказаться более удобным выбор каких--либо других координат.
Любые $s$\, величины $q_1, q_2, \dotsc , q_s$, вполне характеризующие
положение системы (с $s$\, степенями свободы), называется ее
\parbox{146.5pt}{\todo[inline, color=green!40]{\textbf{обобщенными
координатами}}}, а производные $\dot{q_i}$\, --- ее
\parbox{134.5pt}{\todo[inline, color=green!40]{\textbf{обобщенными
скоростями}}}.
Задание значений обобщенных координат еще не определяет, однако
<<механического состояния>> системы в данный момент времени в том смысле,
что оно не позволяет предсказать положение системы в последующие моменты
времени. При заданых значениях координат система может обладать
произвольными скоростями, а в зависимости от значения последних будет
различным и положение системы в следующий момент времени (через бесконечно
малый временной интервал $dt$).


Одновременное же задание всех координат и скоростей полностью определяет,
как показывает опыт, состояние системы и позволяет в принципе предсказать
дальнейшее ее движение. С математической точки зрения это значит, что
заданием всех координат $q$\, и скоростей $\dot{q}$\,  в некоторый момент
времени однозначно определяется также и значение ускорений $\ddot{q}$\, в
этот $\text{момент}^{\text{\pdfmarkupcomment[color=blue,opacity=1.0,
subject={Аннотация},author={Примечание}]{}{%
Для кратности обозначений будем часто условно понимать под q совокупность
всех координат $q_1$, $q_2$, $\dots$, $q_s$ и под $\dot{q}$. аналогично
совокупность всех скоростей}}}$.
Соотношения, связывающие ускорения с координатами и скоростями, называется
\parbox{121.7pt}{\todo[inline, color=green!40]{\textbf{уравнениями
движения}}}. По отношению к функциям $q(t)$\, это ---
дифференциальные уравнения второго порядка, интегрирование которых
позволяет в принципе определить эти функции, т.~е. траектории движения
механической системы.


\subsubsection{Принцип наименьшего действия}
Формулировка закона движения механических систем \tooltip{дается}{16} так называемым
\textcolor{red}{\textbf{принципом}} 
\tikz[baseline, outline/.style={draw=#1,thick,fill=#1!50}]
%{\node[outline=blue,fill=blue!50,anchor=base] {\textbf{наименьшего
{\node[outline=blue, left color=blue!50, draw=black!50,thick] {{\bf
наименьшего действия (принципом Гамильтона)}};}.
Согласно этому принципу каждая механическая система характеризуется
определенной функцией:
$$
L(q_1,\, q_2,\, \dotsc ,\, q_s,\, \dot{q}_1,\, \dot{q}_2,\, \dotsc , 
\dot{q}_s,\, t)
$$
или, в краткой записи, $L(q,\, \dot{q},\, t)$, причем движение системы
удовлетворяет следующему условию.

Пусть в момент времени $t=t_1$\, и\, $t=t_2$\, система занимает
определенные положения; характеризуемые двумя наборами значений
координат $\tensor{q}{^1}$\, и $\tensor{q}{^2}$. Тогда между этими
положениями система движется таким образом, чтобы интеграл:
\begin{equation}\label{mech01}
\displaystyle
S = \int^{t_2}_{t_1} L(q, \dot{q}, t)\, dt
\end{equation}


имел наименьшее возможное $\text{значение}^{\text{%
\pdfmarkupcomment[color=blue,opacity=1.0,subject={Аннотация},author={Примечание}]{}{%
Следует, однако, указать, что в такой формулировке принцип наименьшего
действия не всегда справедлив для всей траектории движения в целом,
а лишь для каждого из достаточно малых ее участков; для всей же траектории
может оказаться, что интеграл (2.1) имеет лишь экстремальное , не обательно
минимальное значение. Это обстоятельство, однако, совершенно не существенно
при выводе уравнений движения, использующем лишь условие экстремальности.}}}$.
Функция $L$\, называется
\parbox{108pt}{\todo[inline, color=green!40]{\textbf{функцией Лагранжа}}}
данной системы, а интеграл \Eq{mech01} ---
\parbox{57pt}{\todo[inline, color=green!40]{\textbf{действием}}}.

Так как \parbox{103.6pt}{\todo[inline, color=green!40]{\textbf{функция Лагранжа}}}
содержит только $q$\, и $\dot{q}$,
но не более высокие производные $\ddot{q}$, $\dddot{q}$, $\dotsc$, является
выражением что механическое состояние полностью определяется заданием
координат и скоростей.


Решим задачу об \tooltip{определении минимума интеграла}{9} \Eq{mech01}. Предположим
сначала, что система обладает всего одной степенью свободы, так что должна
быть определена всего одна функция $q(t)$.


Пусть $q=q(t)$\, есть как раз функция, для которой $S$\, имеет минимум.
Это значит, что $S$, возрастает при замене $q(t)$\, на любую функцию вида:
\begin{equation}\label{mech02}
q(t) + \delta q(t)
\end{equation}
где $\delta q(t)$\, --- функция, малая во всем интервале времени от
$\tensor{t}{_1}$\, до\, $\tensor{t}{_2}$\, называется
\parbox{59pt}{\todo[inline, color=green!40]{\textbf{вариацией}}}
функции $q(t)$; поскольку при $t=t_1$\, и\, $t=t_2$\, все сравниваемые
функции \Eq{mech02} должны принимать одни и те же значения
$\tensor{q}{^1}$\, и $\tensor{q}{^2}$, то должно быть:
\begin{equation}\label{mech03}
\delta q(t_1) = \delta q(t_2) = 0
\end{equation}
Изменение $S$\, при замене $q$\, на $q+ \delta q$\, дается разностью:
$$
\int^{t_2}_{t_1} = L (q + \delta q,\, \dot{q} + \delta\dot{q},\, t)\, dt
- \int^{t_2}_{t_1} L(q,\, \dot{q},\, t)\, dt
$$
Разложение этой разности по степеням $\delta q$\, и $\delta \dot{q}$\,
(в подынтегральном выражении) начинается с членов первого порядка.
Необходимым условием минимальности
$S^{\text{\pdfmarkupcomment[color=blue,opacity=1.0,subject={Аннотация},
author={Примечание}]{}{Вообще - экстремальности }}}$\,
является обращение в нуль
совокупности этих членов; ее называют
\parbox{95pt}{\todo[inline, color=green!40]{\textbf{первой вариацией}}}
(или просто вариацией) интеграла. Таким образом, принцип наименьшего
действия можно записать в виде:
\begin{equation}\label{mech04}
\delta S = \delta \int^{t_2}_{t_1} L (q,\, \dot{q},\, t)i\, dt = 0
\end{equation}
или, произведя варьирование:
$$
\int^{t_2}_{t_1} \left(\frac{\partial L}{\partial q} \delta q +
\frac{\partial L}{\partial \dot{q}} \delta \dot{q}\right) dt = 0
$$
Замечая, что $\delta \dot{q} = \frac{d}{dt} \delta q$, принтегрируем
второй член по частям и получим:
\begin{equation}\label{mech05}
\delta S = \frac{\partial L}{\partial \dot{q}}\, \delta q\,
\bigg\vert^{t_2}_{t_1} + \int^{t_2}_{t_1} \left(
\frac{\partial L}{\partial q} - \frac{d}{dt}
\frac{\partial L}{\partial \dot{q}}\right) \delta q\, dt = 0 
\end{equation}
Но в силу условий \Eq{mech03} первый член в этом выражении исчезает.
Остается интеграл, который должен быть равен нулю при произвольных
значениях $\delta q$. Это возможно только в одном случае, если
подынтегральное выражение тождественно обращается в нуль. Таким образом, мы
получаем уравнение:
$$
\frac{d}{dt} \frac{\partial L}{\partial \dot{q}} - 
\frac{\partial L}{\partial q} = 0
$$
При наличии нескольких степеней свободы в принципе наименьшего действия
должны независимо варьироваться $s$\, различными функциями $q_i(t)$.
Очевидно, что мы получим тогда $s$\, уравнений вида:
\begin{equation}\label{mech06}
\frac{d}{dt} \frac{\partial L}{\partial \dot{q}_i} -
\frac{\partial L}{\partial q_i} = 0 \, ( i = 1,\, 2,\, \dotsc ,\, s)
\end{equation}
Это --- искомое дифференциальное $\text{уравнение}^{\text{%
\pdfmarkupcomment[color=blue,opacity=1.0,subject={Аннотация},
author={Примечание}]{}{%
 В вариационном исчислении, рассматривающем формальную задачу об
 определении экстремумов интегралов вида (2.1), они называются
 уравнениями Эйлера}}}$\, называется
\parbox{109.7pt}{\todo[inline, color=green!40]{\textbf{уравнение Лагранжа}}}.
Если функция Лагранжа данной механической
системы известна, то уравнение \Eq{mech06} устанавливает связь между
ускорениями, скоростями и координатами, т.~е. представляют собой уравнения
движения системы.

Для полного определения движения механической системы необходимо знание
начальных условий, характеризующих состояние системы в некоторый заданный
момент времени, например знание начальных значений всех координат и
скоростей.


Пусть механическая система состоит из двух частей $A$\, и $B$, каждая из
которых, будучи замкнутой, имела бы в качестве функции Лагранжа
соответственно функции $L_A$\, и $L_B$. Тогда в пределе, при разведении
частей настолько далеко, чтобы взаимодействие между ними можно было
пренебречь, лагранжева функция всей системы стремится к пределу:
\begin{equation}\label{mech07}
\lim L = L_A + L_B
\end{equation}
Это свойство
\parbox{74.1pt}{\todo[inline, color=green!40]{\textbf{аддитивности}}}
функции Лагранжа
выражает собой тот факт, что
уравнения движения каждой из невзаимодействующих частей не могут содержать
величины, относящиеся к другим частям системы.

Умножение функции Лагранжа механической системы на произвольную постоянную
само по себе не отражается на уравнениях движения. Отсюда вытекает
неопределенность: функции Лагранжа различных изолированных механических
систем могли бы умножаться на любые различные постоянные. Свойство
аддитивности устроняет эту неопределенность, --- оно допускает лишь
одновременное умножение лагранжевых функций всех систем на одинаковую
постоянную, что сводится просто к естественному произволу в выборе единиц
измерения этой физической величины.\\[5pt]

%\tikz[remember picture, overlay, note/.style={rectangle callout, fill=#1}]
%	\node [note=red!50, opacity=.5, overlay,
%		callout absolute pointer={(1.5,0.2)}] at (1.3,0.6) {%
%			{\bf{\tiny Нажмите для возврата}}};
\tikz[remember picture, overlay]
	\node[rectangle callout,cloud puffs=15,aspect=2.5,cloud puff arc=120,
		shading=ball,text=white,overlay,rounded corners,thick,
		callout absolute pointer={(1.5,0.2)}] at (1.3,0.6)
			{\bf{\tiny Нажмите для возврата}};
\noindent
\hypertarget{end_2}{\hyperlink{end_2_1}{Рассмотрим две функции}}
$L' (q,\, \dot{q},\, t)$\, и
$L(q,\, \dot{q},\, t)$, отличающиеся друг от друга на полную производную по
времени от какой--либо функции координат и времени $f(q,\, t)$:
\begin{equation}\label{mech08}
L' (q,\, \dot{q},\, t) = L(q,\, \dot{q},\, t) +
\frac{d}{dt}\, f(q,\, t)
\end{equation}
Вычисление с помощью этих двух функций интегралы \Eq{mech01} связаны
соотношением:
$$
S' = \int^{t_2}_{t_1} L' (q,\, \dot{q},\, t)\, dt =
\int^{t_2}_{t_1} \frac{df}{dt} dt = S + f(q^2,\, t_2) - f(q^1,\, t_1)
$$
т.~е. отличаются друг от друга дополнительным членом, исчезающим при
варьировании действия, так что условие $\delta S' = 0$\, совпадают с
условием $\delta S=0$, и вид уравнений движения остается неизменным.

Таким образом, функция Лагранжа \tooltip{определена}{13} лишь с точностью
до прибавления к ней полной производной от любой функции координат и времени.

\subsubsection{Принцип относительности Галилея}
Для изучения механических явлений надо выбрать какую--либо
\parbox{85.6pt}{\todo[inline, color=green!40]{\textbf{систему отсчета}}}.
В различных системах отсчета законы движения имеют, различный
вид. Если взять произвольную систему отсчета, то может оказаться, что
законы даже совсем простых явлений будут выглядеть в ней весьма сложно.
Возникает задача отысканя такой системы отсчета, в которой законы механики
выглядели бы наиболее просто.

По отношению к произвольной системы отсчета пространство является
неоднородным и неизотропным. Это значит, что если какое--либо тело не
взаимодействует ни с какими другими телами, то, тем не менее, его различные
положения в пространстве и его различные ориентации в механическом
отношении не эквивалентны. То же самое относится в общем случае и ко
времени, которое будет неоднородным, т.~е. его различные моменты
неэквивалентными. 

Возьмем пример: свободное не подвергающееся внешним воздействиям тело не
могло бы покоится, если скорость тела в некоторый момент времени и равна
нулю, то уже в следующий момент тело начало бы двигаться в некотором
направлении.


Оказывается, что всегда можно найти такую систему отсчета, по отношению к
которой пространство является однородным и изотропным, а время ---
однородным. Такая система называется
\parbox{77.7pt}{\todo[inline, color=green!40]{\textbf{инерциальной}}}.
В этой системе свободное тело, покоящееся в некоторый момент времени,
остается в покое неограниченно долго.


Сделает некоторые заключения о виде функции Лагранжа свободно движущейся
материальной точки в инерциальной системы отсчета. Однородность простанства
и времени означает, что эта функция не может содержать явным образом ни
радиус--вектора $r$\, точки, ни времени $t$, т.~е. $L$\, является функцией
лишь от скорости $v$. В силу же изотропии пространства функция Лагранжа не
может зависеть также и от направления вектора $\vec{\mathrm{v}}$, так что
является функцией лишь от его абсолютной величины, т.~е. от квадрата
$\mathrm{v}^2=v^2$.
\begin{equation}\label{mech08}
L = L (v^2)
\end{equation}
Ввиду независимости функции Лагранжа от $r$\,  имеем
$\frac{\partial L}{\partial r} = 0$, поэтому уравнения Лагранжа
имеют~$\text{вид}^{\text{\pdfmarkupcomment[color=blue,opacity=1.0,
subject={Аннотация},author={Примечание}]{}{%
Под производной скалярной величины по вектору подразумевается вектор,
компоненты которого равны производным от этой величины по соответствующим
компанетам вектора}}}$:
$$
\frac{d}{dt}\frac{\partial L}{\partial v} = 0 
$$
откуда $\frac{\partial L}{\partial v} = \text{константа}$. Но поскольку
$\frac{\partial L}{\partial v}$\, является функцией только от скорости, то
отсюда следует, что и:
\begin{equation}\label{mech09}
\boldsymbol{\mathrm{v}} = \text{константа}
\end{equation}
Приходим к выводу, что в инерциальной системе отсчета всякое свободное
движение происходит с постоянной по величине и направлению скоростью.
Это утверждение составляет содержание так называемого
\parbox{85pt}{\todo[inline, color=green!40]{\textbf{закона инерции}}}. 


Опыт показывает, что законы свободного движения будут одинаковыми в таких
системах, но и что они будут и во всех других механических отношениях
полностью эквивалентны. Существует не одна, а бесконечное множество
инерциальных систем отсчета, движущихся друг относительно друга
прямолинейно и равномерно. Во всех этих системах свойства пространства и
времени одинаковы и одинаковы все законы механики. Это утверждение
составляет содержание
\parbox{181.7pt}{\todo[inline, color=green!40]{\textbf{принципа
относительности Галилея}}}.

Полная механическая эквивалентность всего бесчисленного множества таких
систем показывает в то же время, что не существует никакой одной
<<абсолютной>> системы отсчета, которую можно было бы предпочесть другим
системам.


Координаты $r$\, и $r'$\, одной и тойже точки в двух различных системах
отсчета $K$\, и $K'$, из которых вторая движется относительно первой
со скоростью $V$, связанны друг с другом соотношением:
\begin{equation}\label{mech10}
r = r' + Vt
\end{equation}
При этом подразумевается, что ход времени одинаков в обеих системах
отсчета:
\begin{equation}\label{mech11}
t = t'
\end{equation}
Предположение об абсолютности времени лежит в самой основе представлений
классической \pdfmarkupcomment[color=yellow,opacity=1.0,
subject={Аннотация},author={Примечание}]{механики}{%
Оно не справедливо в механике теории относительности}.


Формулы \Eq{mech10} и \Eq{mech11} называют
\parbox{133.7pt}{\todo[inline, color=green!40]{\textbf{преобразованием
Галилея}}}.
Принцип относительности Галилея можно \tooltip{сформулировать}{13} как
требование инвариантности уравнений движения механики по отношению
к этому преобразованию.

\subsubsection{Функция Лагранжа свободной материальной точки}
Для определения вида функции Лагранжа, рассмотрим простейший случай ---
свободное движение материальной точки относительно инерциальной системы
отсчета. Функция Лагранжа может зависеть лишь от квадрата вектора скорости.
Для выяснения вида этой зависимости воспользуемся принципом
относительносити Галилея. Если инерциальная система отсчета $K$\, движется
относительно инерциальной системы отсчета $K'$\, с бесконечно малой
скоростью $\varepsilon$, то $\mathrm{v}' = \mathrm{v} + \varepsilon$.
Так как уравнения движения во
всех системах отсчета должны иметь один и тот же вид, то функция Лагранжа
$L(v^2)$\, должна при таком преобразовании перейти в функцию $L'$\,
которая если и отличается от $L(v^2)$, то лишь на полную производную от
функции координат и
времени~\hyperlink{end_2}{\hypertarget{end_2_1}{(см.~конец~\P~2)}}.
$$
L'=L(v'^2) = L(v^2 + 2\mathrm{v}\varepsilon + \varepsilon^2)
$$
Разлагая это выражение в ряд по степеням $\varepsilon$\,  и пренебрегая
бесконечно малыми высших порядков, получим:
$$
L(v'^2) = L(v^2) + \frac{\partial L}{\partial v^2} 2\mathrm{v}\varepsilon
$$
Второй член правой части этого равенства будет полной производной по
времени только в том случае, если от зависит от скорости $\mathrm{v}$\,
линейно. Поэтому $\frac{\partial L}{\partial v^2}$\, от скорости не зависит
т.~е. функция Лагранжа в
рассматриваемом случае прямо пропорциональна квадрату скорости:
\begin{equation}\label{mech12}
L = \frac{m}{2}v^2
\end{equation}
где $m$--постоянная.


Из того, что функция Лагранжа такого вида удовлетворяет принципу
относительности Галилея в случае бесконечно малого преобразования скорости,
непосредственно следует, что она удовлетворяет этому принципу и вслучае
конечной скорости $\mathrm{V}$\, системы отсчета $K$\, относительно $K'$.
Действительно:
$$
L'=\frac{m}{2}v'^2 = \frac{m}{2}(\mathrm{v} + \mathrm{V})^2 =
\frac{m}{2}V^2 + 2\frac{m}{2}\mathrm{v}\mathrm{V}
+ \frac{m}{2} V^2
$$
или
$$
L' = L + \frac{d}{dt}\left(2\frac{m}{2}r\mathrm{V} +\frac{m}{2}V^2t\right)
$$
Второй член является производной и может быть опущен.


Величина $m$\, называется
\parbox{41.7pt}{\todo[inline, color=green!40]{\textbf{массой}}}
материальной точки. В силу свойства аддитивности функции Лагранжа,
для системы невзаимодействующих точек \pdfmarkupcomment[color=yellow,
opacity=1.0,subject={Аннотация},author={Примечание}]{имеем}{%
В качестве индекса, указывающего номер частицы, мы будем пользоваться первыми
буквами латинского алфавита, а для индексов, нумерующих координаты, используем
буквы i, k, l, . . .}:
\begin{equation}\label{mech13}
L = \sum_a \frac{m_av^2_a}{2}
\end{equation}
Определение массы приобретает реальный смысл при учете этого свойства.
Всегда можно умножить функцию Лагранжа на любую постоянную: это не
отрожается на уравнениях движения. Для функции \Eq{mech13} такое умножение
сводится к изменению единицы измерения массы; отношения же масс различных
частиц, которые только и имеют реальный физический смысл, остаются при этом
преобразовании неизменными.

Как видно, масса не может быть отрицательной. Согласно принципу наименьшего
действия для действительного движения материальной точки из точки 1
пространства в точку 2  интеграл:
$$
S = \int^{2}_{1} \frac{mv^2}{2} dt
$$
имеет минимум. Если бы масса была отрицательной, то для траекторий, по
которым частица сначала быстро удаляется от 1, а затем быстро приближается
к 2, интеграл действия принимал бы сколь угодно большие по абсолютной
величине отрицательные значения, т.~е. не имел бы минимума.

\pdfmarkupcomment[color=yellow,opacity=1.0,subject={Аннотация},
author={Примечание}]{Необходимо заметить, что}{%
Сделанная в примечании на стр 10. оговорка не мешает этому выводу,
так как при m < 0 интеграл не мог бы иметь минимума ни для какого малого
участка траектории}:
\begin{equation}\label{mech14}
v^2 = \left(\frac{dl}{dt}\right)^2 = \frac{dl^2}{dt^2}
\end{equation}
Поэтому для составления функции Лагранжа достаточно найти квадрат длины
элемента дуги $dl$\, в соответствующей системе координат.\\

\noindent
В декартовых координатах, например, $dl^2=dx^2+dy^2+dz^2$, поэтому:
\begin{equation}\label{mech15}
L = \frac{m}{2}\left(\dot{x}^2 + \dot{y}^2 + \dot{z}^2\right)
\end{equation}
В цилиндрических $dl^2=dr^2+r^2d\varphi^2+dz^2$, откуда:
\begin{equation}\label{mech16}
L = \frac{m}{2}\left(\dot{r}^2 + r^2\dot{\varphi}^2 + \dot{z}^2\right)
\end{equation}
В сферических $dl^2=dr^2+r^2d\theta^2+r^2\sin^2\theta d\varphi^2$\, и:
\begin{equation}\label{mech17}
L = \frac{m}{2}\left( \dot{r}^2 + r^2\dot{\theta}^2 +
r^2\sin^2\theta\dot{\varphi}^2\right)
\end{equation}

\subsubsection{Функция Лагранжа системы материальных точек}
Рассмотрим систему материальных точек, взаимодействующих друг с другом, но
ни с какими посторонними телами; такую систему называют
\parbox{59.7pt}{\todo[inline, color=green!40]{\textbf{замкнутой}}}.
Взаимодействие между материальными точками
может быть описано прибавлением к функции Лагранжа
невзаимодействующих точек \Eq{mech13} определенной
\pdfmarkupcomment[color=yellow, opacity=1.0,subject={Аннотация},
author={Примечание}]{зависящей от характера взаимодействия}{Это утверждение
относится к классической - нерелятивисткой - механике}
функции координат. Обозначим эту функцию --- U:
\begin{equation}\label{mech18}
L = \sum_a \frac{m_av^2_a}{2} - U (r_1,\, r_2,\, \dotsc)
\end{equation}
($r_a$\, --- радиус--вектор $a$--й точки. Это есть общий вид функции
Лагранжа замкнутой системы. Сумму:
$$
T = \sum_a\frac{m_{a}v^2_a}{2}
$$
называют \parbox{120.7pt}{\todo[inline, color=green!40]{\textbf{кинетической
энергией}}}, а функцию
$U$\, ---
\parbox{127.7 pt}{\todo[inline, color=green!40]{%
	\textbf{потенциальной энергией}}} системы.


Тот факт, что потенциальная энергия зависит только от расположения всех
материальных точек в один и тот же момент времени, означает, что  изменение
положения одной из них мгновенно отражается на всех остальных. Неизбежность
такого характера взаимодействий связана с основными предпосылками ---
абсолютностью времени и принципом относительности Галилея.


Если бы взаимодействие распространялось не мгновенно, т~е. с конечной
скоростью, то эта скорость была бы различна в разных (движущихся друг
относительно друга) системах отсчета, так как абсолютность времени
автоматически означает применимость обычного правила сложения скоростей ко
всем явлениям. Тогда законы движения взаимодействующих тел были бы различны
в разных инерциальных системах остчета, что противоречило бы принципу
относительности.

Вид функции Лагранжа \Eq{mech18} показывает, что время не только однородно,
но и изотропно, т.~е. его свойства одинаковы в обоих направлениях.
Замена $t$\, на $-t$\, оставляет функцию Лагранжа, а следовательно, и
уравнения движения неизменными. Если в системе возможно некоторое движение,
то всегда возможно и обратное движение, т.~е. такое, при котором система
проходит те же состояния в обратном порядке. Все движения, происходящие по
законам классической механики, обратимы.

Зная функцию Лагранжа, мы можем составить уравнение движения:
\begin{equation}\label{mech19}
\frac{d}{dt} \frac{\partial L}{\partial v_a} =
\frac{\partial L}{\partial r_a}
\end{equation}

Подставив сюда \Eq{mech18},  получим:
\begin{equation}\label{mech20}
m_a\frac{dv_a}{dt} = - \frac{\partial U}{\partial r_a}
\end{equation}
\tooltip{Уравнения движения}{22} в этой форме называются 
\parbox{117.8pt}{\todo[inline, color=green!40]{\textbf{уравнениями
Ньютона}}}  и
представляют собой основу механики системы взаимодействующих частиц.
Вектор:
\begin{equation}\label{mech21}
F_a = -\frac{\partial U}{\partial r_a}
\end{equation}
стоящий в правой стороне уравнения \Eq{mech20}, называется
\parbox{36.8pt}{\todo[inline, color=green!40]{\textbf{силой}}},
действующей на  $a$-ю точку. Вместе с $U$\, она зависит лишь от координат
всех частиц, но не от их скоростей. Уравнение \Eq{mech20}  показывают
поэтому, что и векторы ускорения частиц являются функциями только от
координат.

Потенциальная энергия есть величина, определяемая лишь с точностью до
прибавления к ней произвольной постоянной. Способ выбора этой постоянной
заключается в том, чтобы потенциальная энергия стремилась к нулю при
увеличении расстояний между частицами.

Если для описания движения используются не декартовы координаты точек, а
произвольные обобщенные координаты $q_i$,  то для получения лагранжевой
функции надо произвести соответствующие преобразование:
$$
x_a = f_a(q_1,\, q_2,\, \dotsc ,\, q_s),\quad
\dot{x}_a = \sum_k \frac{\partial f_a}{\partial q_k} \dot{q}_k\quad
\text{и т.~д.}
$$

Подставляя эти выражения в функцию:
$$
L = \frac{1}{2} \sum_a m_a \left(\dot{x}^2_a + \dot{y}^2_a + \dot{z}^2_a
\right) - U
$$
получим искомую функцию Лагранжа, которая будет иметь вид:
\begin{equation}\label{mech22}
L = \frac{1}{2} \sum_{i,\, k} a_{ik} (q)\dot{q}_i \dot{q}_k - U(q)
\end{equation}
где $a_{ik}$\, --- функции только от координат. Кинетическая энергия в
обобщенных координатах по--прежнему является квадратичной функцией
скоростей, но может зависеть также и от координат.

Рассмотрим теперь незамкнутую систему $A$, взаимодействующую с другой
системой $B$, совершающей заданное в заданном внешнем поле (создаваемом
системой $B$). Поскольку уравнения движения получаются из принципа
наименьшего действия путем независимого варьирования каждой из координат
(т.~е. как бы считая остальные известными), мы можем для нахождения
функции Лагранжа $L_A$\, системы $A$\, воспользоваться лагранжевой
функцией $L$\, всей системы $A+B$, заменив в ней координаты $q_B$\,
заданными функциями времени.

Предполагая систему $A+B$\, замкнутой, будем иметь:
$$
L = T_A (q_A,\, \dot{q}_A) + T_B(q_B,\, \dot{q}_B - U(q_A,\, q_B)
$$
где первые два члена представлюят собой кинетические энергии систем
$A$\, и $B$, а третий член --- их совместную потенциальную энергию.
Подставив вместо $q_B$\, --- заданные функции времени и опустив член
$T(q_B(t)),\, \dot{q}_B(t))$, зависящих только от времени и поэтому
являющийся полной производной от некоторой  другой функции времени),
получим:
$$
L = T_A (q_A,\, \dot{q}_A) - U(q_A,\, q_B(t))
$$
Таким образом, движение системы во внешнем поле описывается функцией
Лагранжа обычного типа с тем лишь отличием, что теперь потенциальная
энергия может зависеть от времени явно.

Так, для движения одной частицы во внешнем поле общий вид функции
Лагранжа:
\begin{equation}\label{mech23}
L = \frac{mv^2}{2} - U(r,\, t)
\end{equation}

и уравнение движения:
\begin{equation}\label{mech24}
m\dot{v} = - \frac{\partial U}{\partial r}
\end{equation}
\parbox{69.8pt}{\todo[inline, color=green!40]{\textbf{Однородным}}}
	называют:\\[3pt]
\tikz[baseline, outline/.style={draw=#1,thick,fill=#1!50}]
{\node[outline=blue,fill=blue!50,anchor=base,draw opacity=0.8,
	fill opacity=0.2,text opacity=1] {\textbf{поле во всех
точках которого на частицу действует одна и та же сила $F$}};}\\[3pt]


Потенциальная энергия в таком поле равна:
\begin{equation}\label{mech25}
U = - Fr
\end{equation}

\noindent\pdfsquarecomment[color=red!30!green,subject={Информация},
height=0.2cm,width=12.8cm,voffset=-0.1cm,hoffset=0.5cm,opacity=0.2,
justification=right,color=CornflowerBlue,bse=cloudy,bsei=2.5,
linewidth=2pt,author={Заключительная информация!}]{{\tiny
Часто приходится иметь дело с такими механическими системами, в которых
взаимодействие между телами (материальными точками) имеет, как говорят,
характер \textbf{связей}, т.~е. ограничений, налагаемых на взаимное
расположение тел. Фактически такие связи осуществляются путем скрепления
тел различными стержнями, нитями, шарнирами и т.~п. Это обстоятельство
вносит в движение новый фактор --- движение тел сопровождается трением
в местах их соприкосновения, в результате чего задача выходит, вообще
говоря, за рамки чистой механики. Однако во многих случаях трение в системе
оказывается настолько слабым, что его влиянием на движение можно полностью
пренебречь. Если к тому же можно пренебречь массами <<скрепляющих
элементов>> системы, то роль последних сведется просто к уменьшению числа
степеней свободы системы $s$ (по сравнению с числом $3N$).
Для определения ее движения можно при этом снова пользоваться функцией
Лагранжа вида \Eq{mech22} с числом независимых обобщенных координат,
отвечающих фактическому числу степеней свободы}.}
\pdffreetextcomment[color=red,subject={Аннотация},height=0.3cm,
width=12cm,voffset=-0.15cm,hoffset=0.9cm,opacity=1.0,justification=right,
type=typewriter,fontsize=7pt,fontcolor=blue,
author={Заключительная информация!}]{%
	В заключении, приведем дополнительное разьяснение по механическим
	системам со связами, смотрите аннотацию.}
