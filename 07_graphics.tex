\newpage
\disableTemplate{Lugatitle}
\AddToTemplate{Lugamathbg}
\AddToTemplate{Navigatorbarbg}
\AddToTemplate{LugatexLogo}

%\begin{tikzpicture}[overlay]
%	\clip (-6.6,-5.8) rectangle ++(3,8);
%	\node (matter) {\mbox{}};
% 
%  	\begin{pgfonlayer}{background}
%  		\clip (-8,-8) rectangle ++(16,12);
%  		\colorlet{upperleft}{green!50!black!25}
%  		\colorlet{upperright}{orange!25}
%  		\colorlet{lowerleft}{red!25}
%  		\colorlet{lowerright}{blue!25}
%  		% The large rectangles:
%  		\fill [upperleft]  (matter) rectangle ++(-8,8);
%  		\fill [upperright] (matter) rectangle ++(8,8);
%  		\fill [lowerleft]  (matter) rectangle ++(-8,-8);
%  		\fill [lowerright] (matter) rectangle ++(8,-8);
%  		% The shadings:
%  		\shade [left color=upperleft,right color=upperright]
%  		([xshift=-1.5cm]matter) rectangle ++(2,20);
%  		\shade [left color=lowerleft,right color=lowerright]
%  		([xshift=-1cm]matter) rectangle ++(2,-20);
%  		\shade [top color=upperleft,bottom color=lowerleft]
%  		([yshift=-1cm]matter) rectangle ++(-20,2);
%  		\shade [top color=upperright,bottom color=lowerright]
%  		([yshift=-1cm]matter) rectangle ++(20,2);
%  	\end{pgfonlayer}
%  
%  \end{tikzpicture}
  
%%%--> \begin{textblock}{1}(-2.7,-2)
%%%--> \includegraphics[scale=1]{physics.pdf}
%%%--> \end{textblock}
%%%--> 
%%%--> \newpage
%\hspace{-15pt}

\colorlet{KernFarbe}{red}
\colorlet{HuelleFarbe}{blue}
%\vspace{35mm}
\begin{textblock}{8}(1.875,1.99)
\begin{tikzpicture}[
        Huelle/.style = {ball color=HuelleFarbe!25},
        Kern/.style = {ball color=KernFarbe!25},
        Proton/.style = {ball color=HuelleFarbe!80},
        Elektron/.style = {ball color=KernFarbe!80},
	scale=1.1
    ]

    \shade[Huelle, circular drop shadow={opacity=1},
    	circular drop shadow={shadow scale=1.15}, shading angle=360] (8,0) circle (2cm);
    \shade[Kern]   (8,0) circle (.7cm);

    % Elektronen
    \foreach \cx/\cy in {8/0,8/.3,8/-.3,8.2/.25,8.2/-.25,%
        8.3/.09,8.3/-.09,7.8/.25,7.8/.-.25,7.7/.09,%
        7.7/-.09,7.5/.09,8.5/-.09,7.6/-.3,8/-.5,8/.5,8.4/.4}{
            \shade[Elektron] (\cx,\cy) circle (.6mm);
    }
    % Protonen
    \foreach \cx/\cy in {6.9/-.1,7.4/1.2,7/.7,7.9/1.5,9.2/0.2,%
        9/1,7.5/-1.2,7/-0.8,9/-0.8,8.7/-1.2,7.7/-1.5,8.9/.5, %8.6,-1
        7/1.3,6.6/.4,9.6/-.31,9.6/.6,8.3/1.2}{
            \shade[Proton] (\cx,\cy) circle (.6mm);
    }
    
	\draw[<-] (8.25,.55) -- +(1.7,1) node[above] {\begin{shadowblockb}{1.18cm}
			{\bf \tiny атомное ядро}\end{shadowblockb}};
	\draw[<-] (7,.4) -- +(-2,1) node[above] {\begin{shadowblockb}{1.54cm}
			{\bf \tiny атомная оболочка}\end{shadowblockb}};
	\draw[<-] (6.8,-.085) -- +(-3,.4) node[above]
	{\begin{shadowblockb}{0.75cm}
			{\bf \tiny электрон}\end{shadowblockb}};
		\draw[<-] (7.5,-.35) -- +(-3,-.7) node[below] {\begin{shadowblockb}{0.6cm}
			{\bf \tiny протон}\end{shadowblockb}};
\end{tikzpicture}
\end{textblock}

\begin{textblock}{9}(3.3,6)
	\tikz \node {{\bf {\tiny 75\%}}};
\end{textblock}

\begin{textblock}{10}(1.2,7.21)
	\tikz \node {{\bf {\tiny 20\%}}};
\end{textblock}

\begin{textblock}{11}(0.1,6.5)
	\tikz \node {{\bf {\tiny 4\%}}};
\end{textblock}

\begin{textblock}{12}(1,5.7)
	\tikz \node {{\bf {\tiny 1\%}}};
\end{textblock}

\begin{center}
	\vglue 3pt
\begin{tikzpicture}[scale=0.85]
  \path[mindmap,concept color=yellow, text=white,
  	every node/.style={concept, circular drop shadow,execute at begin
  		node=\hskip0pt, level distance=3cm, sibling angle=90}]
    node[concept, concept color=yellow, line width=1ex, text=white,
		 fill=black, font=\large\scshape, circular drop shadow,execute at
		 begin node=\hskip0pt] (concept) {{\bf Материя}}
    [clockwise from=-60]
    child[concept color=green!50!black] {
	node[concept, line  width=1ex, fill=magenta!10,
		text=black]
		{{\bf Не имеет\\ Структуру}}
	[clockwise from=300]
	child { node[concept] {Энергия, ЭМ Поле}
	  [clockwise from=0]
	  child [concept color=black!50!magenta!90 ]{ node[concept] {\scalebox{0.9}{Излучение}}}
	 	 child [sibling angle=180, concept color=black!80!blue!90] { node[concept] {Скрытая\\ энергия}}}
	  }
    child[concept color=blue] {
      node[concept, concept color=blue, line  width=1ex, fill=magenta!10,
	  	text=black]
		{{\bf Имеет\\ Структуру}}
	  [clockwise from=-120]
	  child { node[concept] {Вещество}
	  [clockwise from=0]
	  child [concept color=black!80!blue!90] { node[concept] {Скрытое\\ вещество}}
	  child [sibling angle=180, concept color=blue!70!cyan!50, text=black] { node[concept] {Обычное\\ вещество}}}
    };
\end{tikzpicture}
\end{center}

\begin{textblock}{5}(10.65,2.5)
       \begin{shadowblock}{4cm}
%	       \parbox{110pt}{%
		\begin{tikzpicture}
		\node[text width=5cm] {%
		{\tiny{\bf Примеры диаграмм со слоями в}
			$\text{{\tiny{\bf формате}}}^{\text{%
		\pdfcomment[color={1 0.3 1},voffset=4pt,opacity=0.45,
		subject={Аннотация},author={Просмотр вложенных документов}]{%
		Для просмотра демонстрационной версии LugaTeX}}}$\\[-2pt]
		\hspace{5pt}\pdfmarkupcomment[color=white,opacity=1.0,
		subject={Аннотация},author={Просмотр вложенных PDF документов}]{}{Теоретическая физика}
		\hspace{1pt} Ядерный реактор PWR:
			\hspace{1.1pt}{\bf нажмите}~\attachfile[icon=PushPin,scale=0.4,
	       color=1 0 0,mimetype=application/pdf,
	       author={lugatex@yahoo.com}]{ocg_layers_05.pdf}\\[2pt]
	       \includegraphics[scale=0.43]{pressurized-water-reactor}\\
	       \hspace{5pt}\pdfmarkupcomment[color=white,opacity=1.0,
		   subject={Аннотация},author={Просмотр вложенных PDF документов}]{}{Теоретическая физика}
		   \hspace{1pt} Теоремы и определения:~{\bf нажмите}~\attachfile[icon=PushPin,scale=0.4,
			color=1 0 0,mimetype=application/pdf,
				author={lugatex@yahoo.com}]{ocg_layers_01.pdf}\\[-5pt]
		\hspace{5pt}\pdfmarkupcomment[color=white,opacity=1.0,
		subject={Аннотация},author={Просмотр вложенных PDF документов}]{}{Теоретическая физика} 
		\hspace{1pt} Коммутативная алгебра:\hspace{4.5pt}{\bf %
			нажмите}~\attachfile[icon=PushPin,scale=0.4,color=1 0 0,mimetype=application/pdf,
				author={lugatex@yahoo.com}]{ocg_layers_02.pdf}}};

		\end{tikzpicture}
       \end{shadowblock}
\end{textblock}

\begin{textblock}{5}(0,3.15)
\newcommand{\slice}[4]{
  \pgfmathparse{0.4*#1+0.35*#2}
  \let\midangle\pgfmathresult

  % slice
  \draw[
  	thick,
	inner color=white,
	top color=light-blue,
  	bottom color=black!70!blue,
	draw=magenta,
	bottom color=white,
	%circular drop shadow,
	circular drop shadow={shadow scale=1.05},
	circular drop shadow={opacity=0.15},
	%shading=axis,
	shading angle=6,
	rotate=180,
	line width=0.1ex] (0,0) -- (#1:1) arc (#1:#2:1) -- cycle;

  % outer label
%  \node[label=\midangle:#4] at (\midangle:1) {};

  % inner label
  \pgfmathparse{min((#2-#1-10)/110*(-0.3),0)}
  \let\temp\pgfmathresult
  \pgfmathparse{max(\temp,-0.5) + 0.8}
  \let\innerpos\pgfmathresult
%  \node at (\midangle:\innerpos) {#3};
}

\begin{tikzpicture}[scale=2.2]

\newcounter{a}
\newcounter{b}
\foreach \p/\t in {1/\mbox{}, 4/\mbox{}, 20/\mbox{}, 75/\mbox{}}
  {
    \setcounter{a}{\value{b}}
    \addtocounter{b}{\p}
    \slice{\thea/100*360}
          {\theb/100*360}
		  {{\tiny \p\%}}{\t}
  }
\end{tikzpicture}
	
\end{textblock}

\begin{textblock}{8}(0.5,10.15)
	\noindent
	{\tiny{\bf 
	\textcolor{red!50!yellow!90}{\underline{Материя во Вселенной}:}\\
	\textcolor{black!80!blue!90}{Скрытая энергия} "--- 75\%\\
	\textcolor{black!80!blue!90}{Скрытое вещество} "--- 20\%\\
	\textcolor{blue!70!cyan!50}{Обычное вещество} "--- 4\%\\
	\textcolor{black!50!magenta!90}{Излучение} "--- 1\%
	}}
\end{textblock}

%\tikz \draw[brown,overlay] (-0.6,1.87) -- (-0.9,1.87) -- (-0.9,5) circle (2pt);
\tikz \draw[brown,overlay, ->] (-0.6,1.87) -- (-0.9,1.87) -- (-0.9,6.3) --
	(-0.6,6.34);
\tikz \draw[brown,overlay, ->] (-0.72,2.27) -- (-0.88,2.27) -- (-0.88,5.4) --
	(-0.88,5.8) -- (-0.65,5.9);
\tikz \draw[brown,overlay, ->] (-0.85,2.73) -- (-0.85,2.73) -- (-0.85,4) --
	(-0.1,4.7);
\tikz \draw[brown,overlay, ->] (1.2,3.13) -- (2.5,3.13) -- (2.5,4.3);

%Привет пробная часть срытого текста, \xBld{mythoughts}the number is the
%value of the integral $\int_0^4 4x + \frac14\,dx$. \eBld Want a
%hint? \setLinkText[%
%\A{\JS{%
%	var oLayer = getxBld(``mythoughts'');
%	if ( oLayer != null )
%		oLayer.state = !oLayer.state;
%		}}
%]{\textcolor{red}{Click here}} I hope that hint worked for you.
%Click on the link to hide the layer again.\par
%

% Navigational Panel 
% \newpage
% \includepdf{adobe_navibar.pdf}
