\newpage	%--- page 7
\disableTemplate{LugatexLogo}
\disableTemplate{Navigatorbarbg}
\disableTemplate{Lugapdecover}
\AddToTemplate{Lugamathbg}
\AddToTemplate{Navigatorbarbg}
\AddToTemplate{LugatexLogo}

\section{Уравнения математической физики}
\subsection{Понятие решения и постановка задачи}
Многие \tooltip{физические процессы}{6} в таких областях науки и техники,
как \tooltip{механика}{7},
теплофизика, \tooltip{электричество и магнетизм}{8}, оптика,
описывается с помощью \tooltip{уравнений}{6} с частными производными.
Большенство уравнений самой
\tooltip{математической физики}{17} есть уравнения с частными производными.
В отличие от
обыкновенных дифференциальных уравнений, в которых неизвестная функция
зависит только от одной переменной, в уравнениях с частными производными
неизвестная функция зависит от нескольких переменных (например,
напряженность электрического поля
$\displaystyle \vec{E}(x,\, y,\, z,\, t)$\, зависит от
пространственных координат $\displaystyle x,\, y,\, z$\, и времени
$t$).

\tikz[baseline, outline/.style={draw=#1,thick,fill=#1!50}]
{\node[outline=blue, left color=blue!50, draw=black!50,thick] {{\bf %
К основным уравнениям с частными производными относятся:}};}

%%%---> \shade[ball color=green] (9,.5) circle (.5cm);
%%%---> \tikz[baseline] \shadedraw [shading=ball] (0,0) circle (1mm) ;
\begin{itemize}
\item[] \tikz[baseline] \node[ball color=green,circle,text=black] {1};\quad
	\tikz[baseline] \node 	{трехмерное уравнение Лапласа;};
$$
\displaystyle \Delta u =0,\quad \text{или}\quad u_{xx}+u_{yy}+u_{zz} =0
$$
\item[] \tikz[baseline] \node[ball color=green,circle,text=black] {2};\quad
	\tikz[baseline] \node   {волновое уравнение;};
$$
\displaystyle u_{tt} = a^2\Delta u
$$
\item[] \tikz[baseline] \node[ball color=green,circle,text=black] {3};\quad
	\tikz[baseline] \node   {уравнение теплопроводности;};
$$
\displaystyle u_t = a^2\Delta u
$$
\end{itemize}
В этих уравнениях неизвестная функция
$u=u(x,\, y,\, z)\, (u=u(x,\, y,\,z,\, t))$\, зависит от нескольких
переменных. Количество независимых переменных определяется размерностью
пространства, в котором происходит физическое явление, и временной
переменной (в случае нестационарных явлений). Широкий класс образует
линейные уравнения с частными производными относительно функции, зависящей
от двух переменных. Эти уравнения можно записать в виде
\begin{equation}\label{pde01}
\displaystyle
a_{11}u_{xx} + 2a_{12}u_{xy} + a_{22}u_{yy} + b_1u_x +
b_2u_y + cu = f
\end{equation}


где $\displaystyle a_{11},\, a_{12},\, a_{22},\, b_1,\, b_2,\, c,\, f$\,
--- заданные функции независимых переменных $x$\, и $y$; $u$\, ---
неизвестная функция. Все линейные уравнения с частными производными
второго порядка вида \Eq{pde01} при условии,
что $\displaystyle a^{2}_{11}+a^2_{12}+a^2_{22} \ne 0$\,  в точке
$(x,\, y)$, относятся к одному из трех типов:
\begin{itemize}
\item[] \tikz \shadedraw [shading=ball] (0,0) circle (2mm) ;\quad
\tikz \node {эллиптическому;};
\item[] \tikz \shadedraw [shading=ball] (0,0) circle (2mm) ;\quad
\tikz \node {гиперболическому;};
\item[] \tikz \shadedraw [shading=ball] (0,0) circle (2mm) ;\quad
\tikz \node {параболическому (в этой же точке $(x,\, y)$;)};
\end{itemize}


Уравнения эллиптического типа описывают стационарные процессы и
определяются условием $a^2_{12}-a_{11}a_{22} < 0$.\\

Уравнения гиперболического типа описывают волновые процессы и определяются
условием $a^2_{12}-a_{11}a_{22}>0$.\\

Уравнения параболического типа описывают процессы распространения тепла,
диффузии (и некоторые другие) и определяются условием
$a^2_{22}-a_{11}a_{22}=0$.\\

Заметим, что в уравнениях гиперболического и параболического типов
переменная $t$\, носит характер временной переменной.

Таким образом, уравнения $u_{xx}+u_{yy}=0$,\, и $u_{tt}=a^2u_{xx}$,\,
$u_t=a^2u_{xx}$\, есть соответсвенно уравнения эллиптического,
гиперболического и параболического типа.\\[5pt]

\noindent
\tikz[baseline] \node[left color=red!50, draw=black!50, fill=red!50, thick]
{{\bf Понятия решения}};\\


Решением (классическим) дифференциального уравнения с частными
производными называется функция (обладающая производными, входящими
в уравнение), которая при подстановке в уравнение обращается в его
тождество по независимым переменным в
рассматриваемой области. Например, функция $u(x, t)=\sin x\sin(at)$\, ---
решение уравнения $a^2u_{xx}-u_{tt}=0$\, в области
$-\infty < x, t < \infty$. Отметим, что у одного и того же уравнения
существует много разных решений. При изучении обыкновенных дифференциальных
уравнений вы уже сталкивались с подобными решениями. В случае же уравнения
с частными производными множество решений значительно шире. Например,
множество решений уравнений $y''=0$\, дается формулой $y=C_1x+C_2$,
где $C_1$\, и $C_2$\, --- произвольные постоянные, а множество решений
уравнения $a^2u_{xx} =u_{tt}$\, дается формулой
$u(x, t)= f (x-at)+\text{\textg}(x+at)$, где $f$\, и \textg --- произвольные дважды
дифференцируемые функции.


Приведенное выше понятие решения не является единственно возможным. В
настоящее время широко развита теория так называемых обобщенных
решений.\\[5pt]

\noindent
\tikz[baseline] \node[left color=red!50, draw=black!50, fill=red!50, thick]
{{\bf Постановка задачи}};\\


Для того чтобы выделить единственное решение
из множества решений, необходимо задать дополнительные условия. Эти условия
бывают разных видов --- в зависимости от типа уравнений. Другими словами,
задача --- это уравнение с дополнительными условиями.

Для нестационарных процессов, изучаемых во всем пространстве, необходимо
задавать начальные условия. В этом случае приходим к задаче Коши. Типичными
примерами этой задачи являются следующие задачи:
\begin{equation*}
\begin{cases}
u_{xx} = a^2 u_{xx}, \quad  -\infty < x < + \infty, \quad  t>0\\
u(x, 0)  = f(x), \quad u_t(x, 0)=\text{\textg}(x), \quad -\infty < x < + \infty
\end{cases}
\end{equation*}

\begin{equation*}
\begin{cases}
u_t = a^2 u_{xx}, \quad -\infty < x < + \infty, \quad t>0\\
u(x, 0) = f(x), -\infty < x < + \infty
\end{cases}
\end{equation*}

\begin{equation*}
\begin{cases}
u_{tt} = a^2\Delta u, \quad -\infty x, y, z < +\infty, \quad t>0\\
u(x, y, z, 0) = f(x, y, z) \quad - \infty < x, y, z < +\infty\\
u_t(x, y, z) 0) = \text{\textg}(x, y, z)
\end{cases}
\end{equation*}

\begin{equation*}
\begin{cases}
u_t = a^2 \Delta u, \quad - \infty < x, y, z < + \infty, \quad t>0\\
u(x, y, z, 0) = f(x, y, z),\quad -\infty < x, y, z < +\infty
\end{cases}
\end{equation*}
Задача Коши уже имеет единственное решение ( при естественных условиях).


Если же физический процесс рассматривается в ограниченной области
пространства, то приходим к краевой задаче для стационарных явлений и
смешанной задаче для нестационарных явлений. Например, при изучении
колебаний закрепленной на концах струны получаем смешанную задачу:
\begin{equation*}
\begin{cases}
u_{tt} = a^2 u_{xx}, \quad 0 < x < l, \quad 0 < t < \infty\\
u(0, t) = 0, \quad u(l, t)=0, \quad 0 \leqslant t < +\infty,\\
u(x, 0) = f(x),\quad u_t(x, 0) = \text{\textg}(x), \quad 0\leqslant x \leqslant l
\end{cases}
\end{equation*}
Возможны и другие виды граничных условий.

Аналогично ставится краевая задача для волнового уравнения в трехмерном
случая ($\Omega$\, --- ограниченная область с границей $\partial \Omega$):
\begin{equation*}
\begin{cases}
u_{tt} = a^2 \Delta u, \qquad \text{в цилиндре}\qquad \Omega \times (0, t)\,
\text{при}\, t>0\\
u = 0, \qquad \text{на боковой поверхности}\qquad \partial \Omega \times (0, t)\\
u = f(x, y, z),\quad u_t = \text{\textg}(x, y, z)\\
\qquad \text{на нижнем основании}\qquad
(t=0)\qquad \text{цилиндра}
\end{cases}
\end{equation*}
Далее, процесс распространения тепла в стержне длины
$l$\, описывается одномерным уравнением теплопроводности:
$$
u_{tt} = a^2 u_{xx},\qquad 0<x<l,\qquad 0<t<\infty
$$


где $u(x, t)$\, --- температура стержня в момент времени
$t$\, в точке $x$. Будем считать, что температура на концах
$x=0$\, и $x=l$\, все время поддерживается равной нулю, а вначальный момент
задан температурный профиль $\varphi(x)$. Тогда получаем смешанную задачу
для уравнения теплопроводности:
\begin{equation*}
\begin{cases}
u_t = a^2 u_{xx}, \qquad 0<x<l,\qquad 0<t<\infty\\
u(0, t) = 0, \qquad u(l, t)=0,\qquad t\geqslant 0\\
u(x, 0) =\varphi(x), \qquad 0\leqslant x \leqslant l
\end{cases}
\end{equation*}
Вместо задания температуры на концах (границе) стержня можно задавать
температуру окружающей среды  или тепловой поток  через границую.

Краевая задача для уравнения теплопроводности в трехмерном случае  ставится
аналогично:
\begin{equation*}
\begin{cases}
u_t = a^2 \Delta u \qquad \text{в цилиндре}\qquad
\Omega \times (0, t)\qquad \text{при}\qquad t>0\\
u =0 \qquad \text{на боковой поверхности}\qquad
\partial \Omega\times(0, t)\\
u=f(x, y, z)\qquad \text{на нижнем основании}\qquad (t = 0)
\qquad \text{цилиндра}
\end{cases}
\end{equation*}


Помимо физических явлений, развивающихся в пространстве  и во
времени, существует много явлений, которые не изменяются  с течением
времени. Эти явления в большенстве случаев описываются  краевыми
эллиптическими задачами. В отличие от гиперболического  волнового
уравнения или параболического уравнения теплопроводности,
эллиптические  краевые задачи не требуют начальных условий.
Для них нужны только граничные (краевые) условия. Наиболее важны три типа
граничных условий:

\begin{itemize}
\item[] \tikz[baseline] \node[ball color=cyan,circle,text=black] {1};\quad
\tikz[baseline] \node {граничное условие первого рода
	(условие Дирихле);};
\item[] \tikz[baseline] \node[ball color=magenta,circle,text=black] {2};\quad
	\tikz[baseline] \node {граничное условие второго рода (условие
	Неймана);};
\item[] \tikz[baseline] \node[ball color=brown,circle,text=black] {3};\quad
	\tikz[baseline] \node {граничное условие третьего рода (условие
	Робэна).};
\end{itemize}


Например, краевая задача с граничными условиями первого рода (задача
Дирихле) для уравнения Лапласа ставится так: требуется найти решение
уравнения $\Delta u =0$\, в некоторой области пространства (плоскости),
принимающее на границе заданные значения. В качестве примера можно привести
задачу о нахождении стационарного распределения температуры внутри области
$\Omega$, если задана температура на ее границе $\partial\Omega$.
Другой пример: найти распределение электрического потенциала внутри
области, если известен потенциал на ее границе. Математическая модель
обоих явлений:
\begin{equation*}
\begin{cases}
\Delta u  = 0 \qquad \text{в области}\qquad \Omega\\
u  = \varphi \qquad \text{на границе}\qquad
\partial\Omega\qquad \text{цилиндра}
\end{cases}
\end{equation*}


где $\varphi$\, --- заданная функция.

Краевая задача с граничными условиями второго рода (задача Неймана) для
уравнения Лапласа ставится так: требуется найти решение уравнения в
некоторой области пространства (плоскости), на границе которой задана
внешняя нормальная производная $\frac{\partial u}{\partial n}$\,
(пропорциональная втекающему потоку тепла, вещества). Эта задача и для
стационарной теплопроводности, и для электростатики, если на границе
задан поток, записывается так:
\begin{equation*}
\begin{cases}
 \Delta u  = 0 \qquad \text{в области}\qquad \Omega\\
 \frac{\partial u}{\partial n}  = \varphi \qquad
 \text{на границе}\qquad \partial\Omega
\end{cases}
\end{equation*}
В отличие от задачи Дирихле для уравнения Лапласа задача Неймана имеет
смысл только в том случае, когда полный поток через границу
$\partial\Omega$\, равен нулю, т.~е.
$\int_{\partial\Omega} \frac{\partial u}{\partial s} ds =0$.


Например, внутренняя задача Неймана в единичном круге:
\begin{equation*}
\begin{cases}
 \Delta u  = 0 \quad 0 \leqslant \rho < 1, \quad 0\leqslant \varphi <
 2\pi\\
 \frac{\partial u}{\partial \rho}(1, \varphi)  = 1,\quad 0\leqslant \varphi
 \leqslant 2\pi
\end{cases}
\end{equation*}


не имеет физического смысла, поскольку постоянный единичный поток внутри
области не может обеспечить стационарность решения.


Аналогично  ставятся краевые задачи Дирихле и Неймана для уравнения
Пуассона $\Delta u =f$. Отметим только, что для того, чтобы существовало
решение задачи Неймана:
\begin{equation*}
\begin{cases}
\Delta u  = f \qquad \text{в области}\qquad \Omega\\
\frac{\partial u}{\partial n} = \varphi\qquad \text{на границе}
\qquad \partial\Omega
\end{cases}
\end{equation*}
необходимо и достаточно, чтобы
$\int_\Omega f dx = \oint_{\partial \Omega} \varphi ds$. Другой
особенностью задачи Неймана для уравнения Пуассона, отличающей ее от других
граничных задач, является неединственность решения.


Краевая задача с граничными условиями третьего рода (задача Робэна) для
уравнения Пуассона ставится так: требуется найти решение $u(M)$\, уравнения
в некоторой области $\Omega$\, пространства (плоскости), удовлетворяющее на
границе $\partial\Omega$\, условию
$\frac{\partial u}{\partial n} + \sigma u = \varphi$\, где $\sigma$\, и
$\varphi$\, --- заданные функции на $\partial\Omega$. Эта задача
записывается так:
\begin{equation*}
\begin{cases}
\Delta u  = f \qquad \text{в области}\qquad \Omega\\
\frac{\partial u}{\partial n} + \sigma u = \varphi\qquad
\text{на границе}\qquad \partial\Omega
\end{cases}
\end{equation*}


На разрешимость этой задачи существенно влияет поведение функции $\sigma$,
в частности, ее знак.

\subsection{Дифференциальные уравнения первого порядка, линейные
относительно частных производных}

Рассмотрим дифференциальное уравнение: 
\begin{equation}\label{pde02}
X \frac{\partial z}{\partial x} + Y \frac{\partial z}{\partial y} = Z
\end{equation}
где $X, Y$\, и $Z$\, --- функции $x, y$\, и $z$.

Предварительно решим систему обыкновенных дифференциальных уравнений:
$$
\frac{dx}{Y}=\frac{dy}{Y}=\frac{dz}{Z}
$$
При решении этой системы определяется равенствами:
$$
\omega_1(x, y, z) = C_1
$$
$$
\omega_2(x, y, z) = C_2
$$
Тогда общий интеграл дифференциального уравнения \Eq{pde02}
имеет вид:
$$
\Phi \left[ \omega_1(x, y, z), \quad \omega_2(x, y, z)\right] =0
$$


где $\Phi(\omega_1, \omega_2)$\, --- произвольная
непрерывно--дифференцируемая функция.
