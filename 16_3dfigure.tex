\newpage
\disableTemplate{Lugalibrary}
\AddToTemplate{Lugaart}

\noindent
\hspace{70pt}
\parbox{320pt}{%
\begin{center}
Ипполита Мария Сфорца --- Ippolita Maria Sforza\\
(18 апреля 1446 — 20 августа 1484)\\
Герцогиня Калабрийская, первая жена
Альфонса Калабрийского,\\
будущего короля Неаполя Альфонса II.
\end{center}
}
\vglue 10pt
\noindent
\hspace{60pt}
\parbox{190pt}{%
Родилась в Кремоне. Старшая дочь герцога Миланского Франческо Сфорца
(23 июля 1401 -- 6 марта 1466) и Бьянки Марии Висконти, последней
представительницы рода Висконти (31 марта 1425 -- 28 октября 1468).
Получила отличное образование. Философию и греческий язык ей преподавал
византийский учёный и филолог, обосновавшийся при Миланском дворе,
Константин Ласкари. Известно, что на Соборе в Мантуе (1459) 14--ти
летняя Ипполита выступила с приветственной речью, обращённой к папе Пию II.}
\hspace{2pt}
\parbox{150pt}{%
%\begin{center}
\includemovie[
	poster,
	%toolbar, %same as `controls'
	label=lau.u3d,
	text=(lau.u3d),
	3Daac=60, 3Droll=0, 3Dc2c=0 745.639 0, 3Droo=745.639, 3Dcoo=22.607 -1618.68 -149.864,
	3Dlights=CAD,
]{150pt}{150pt}{Laurana.u3d}\label{ex3d}
%\end{center}
}
\mbox{}\par
\vglue 5pt
\noindent
\hspace{60pt}
\parbox{340pt}{%
\scriptsize{
Существует значительное количество портретов Ипполиты Марии, начиная
с младенческого возраста. Один из наиболее знаменитых --- бюст работы
скульптора Франческо Лаураны. Это несколько идеализированный образ
юной женщины. Впрочем, некоторые историки искусства полагают,
что это портрет Элеоноры Арагонской.}}
\begin{center}
\hspace{50pt}\mbox{\tiny{Интерактивный 3D обьект встроенный в формат PDF}.}
\end{center}
