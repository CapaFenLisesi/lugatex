\documentclass{beamer}

\usepackage{tikz}
\usepackage{ocg}
\usetikzlibrary{shadows}
\usepackage{hyperref}
%----------------------------------------------------------------%

% Command to toggle a layer on/off
\newcommand{\ToggleLayer}[2]{%
	% #1: layer name,
	% #2: link text
	\leavevmode%
	\pdfstartlink user {
		/Subtype /Link
		/Border [0 0 0]%
		/A <<
			/S/JavaScript
			/JS (
				var aOCGs = this.getOCGs();
				for(var i=0; aOCGs && i<aOCGs.length;i++)
				{
					if(aOCGs[i].name == "#1")
					aOCGs[i].state = !aOCGs[i].state;
				}
				)
			>>
		}#2%
		\pdfendlink%
}

% Show all layers on the current page
\newcommand{\ShowAllLayers}[1]{%
	% #1: layer name,
	% #2: link text
	\leavevmode%
	\pdfstartlink user {
		/Subtype /Link
		/Border [0 0 0]%
		/A <<
			/S/JavaScript
			/JS (
				var aOCGs = this.getOCGs();
				for(var i=0; aOCGs && i<aOCGs.length;i++)
				{
				aOCGs[i].state = true;
				}
				)
			>>
		}#1%
		\pdfendlink%
	}

% Hide all layers on the current page
\newcommand{\HideAllLayers}[1]{%
	% #1: link text,
	\leavevmode%
	\pdfstartlink user {
		/Subtype /Link
		/Border [0 0 0]%
		/A <<
			/S/JavaScript
			/JS (
				var aOCGs = this.getOCGs();
				for(var i=0; aOCGs && i<aOCGs.length;i++)
				{
					aOCGs[i].state = false;
				}
				)
			>>
		}#1%
		\pdfendlink%
}

\begin{document}
%%%%%%%%%%%%%%%%%%%%%%%%%%%%%%%%%%%%%%%%%%%%%%%%%%%%%%%%%%%%%%%%%%
\begin{frame}
\centering

% Example by Dominique Würtz
% http://www.texample.net/tikz/examples/intersecting-rings/

% Define the rings. Store them in macros to make things
% more flexible.
\def\ringa{(-1,0) circle (2) (-1,0) circle (3)}
\def\ringb{(1,0) circle (2) (1,0) circle (3)}

\begin{tikzpicture}[note/.style={draw,fill=yellow, drop shadow}]
	% First we fill the intersecting area
	% The \clip command does not allow options, therefore
	% we have to use a scope to set the even odd rule.
	\begin{scope}[even odd rule]
		\begin{ocg}{Intersection}{intersection}{1}
			% Define a clipping path. All paths outside ringa will
			% be cut because the even odd rule is set.
			\clip \ringa;
			% Fill ringb. Since the even odd rule is set, only the
			% ring will be filled, not the hole in the middle.
			\fill[fill=orange] \ringb;
		\end{ocg}
	\end{scope}
	% Then we draw the rings
	\begin{ocg}{Ring A}{ringa}{1}
		\draw \ringa;
	\end{ocg}
	\begin{ocg}{Ring B}{ringb}{1}
		\draw \ringb;
	\end{ocg}
	\begin{ocg}{callout}{callout}{0}
		\draw (0,2) -- ++(-20:3)
			node[note,text width=3cm] {{This is a call out box.
				Click the call out link to close}};
	\end{ocg}

	% Insert a link to toggle the call out.
	\node[blue] {\ToggleLayer{callout}{Toggle call out}};
\end{tikzpicture}

% Insert links to toggle layer visibility
Toggle: \ToggleLayer{Ring A}{ring A} |
\ToggleLayer{Ring B}{ring B} |
\ToggleLayer{Intersection}{Intersection} |
\ShowAllLayers{Show all layers} |
\HideAllLayers{Hide all layers}
\end{frame}
\end{document}
