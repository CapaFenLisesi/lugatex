\documentclass{article}
\usepackage{animate}
\usepackage{tikz}
\usepackage{hyperref}

\makeatletter
% command to create toggle link to animation frame (push button)%
\newcommand{\ShowHideFrame}[3]{%
% #1: anim No. (zero-based),
% #2: frame No. (zero-based),
% #3: link text
\leavevmode%
\pdfstartlink user {
/Subtype /Link
/Border [\@pdfborder]%
/A <<
/S/JavaScript
/JS (
\if@anim@useocg%
if(a#1.fr[#2].state&=& true){
a#1.fr[#2].state=false;
}else{
a#1.fr[#2].state=true;
}
\else
if (a#1.fr[#2].display&=& display.visible){
a#1.fr[#2].display=display.hidden;
}else{
a#1.fr[#2].display=display.visible;
}
this.dirty=false; %reset document status to \upsilonnchanged'
\fi
)
>>
}#3%
\pdfendlink%
}
% command to create link that goes to particular frame while hiding
% the current one (radio button)
\newcommand{\GoToFrame}[3]{%
% #1: anim No. (zero-based),
% #2: frame No. (zero-based),
% #3: link text
\leavevmode%
\pdfstartlink user {
/Subtype /Link
/Border [\@pdfborder]%
/A <<
/S/JavaScript
/JS (
\if@anim@useocg
a#1.fr[a#1.idx].state=false;
a#1.idx=#2;
a#1.fr[#2].state=true;
\else
a#1.fr[a#1.idx].display=display.hidden;
a#1.idx=#2;
a#1.fr[#2].display=display.visible;
this.dirty=false;
\fi
)
>>
}#3%
\pdfendlink%
}

\makeatother

\begin{document}

\begin{center}
\begin{animateinline}[width=0.6\linewidth,controls]{12}

	\multiframe{41}{n=0+1}{

\begin{tikzpicture}
\clip (-0.5,-1.5) rectangle (5,2);
%--------
\draw[thin,->] (-2mm,0) -- (5,0) node[below left] {$t$};
\draw[thin,->] (0,-1.5) -- (0,2) node[above left,rotate=90]
{$y=\sin(t)$};
%--------
\draw[dashed]  (0,0) sin (1,1) cos (2,0) sin (3,-1) cos (4,0);
%--------
\draw[fill=blue]    (\n/10,{sin(\n*9)}) circle (1mm);
%--------
\end{tikzpicture}
}
\end{animateinline}\\[2ex]

\makebox[0.6\linewidth][r]{toggle frame:

\ShowHideFrame{0}{0}{$\sin 0$}
\ShowHideFrame{0}{10}{$\sin \frac{\pi}{2}$}
\ShowHideFrame{0}{20}{$\sin \pi$}
\ShowHideFrame{0}{30}{$\sin \frac{3}{2}\pi$}
\ShowHideFrame{0}{40}{$\sin 2\pi$}}\\
\makebox[0.6\linewidth][r]{go to frame:
\GoToFrame{0}{0}{$\sin 0$}
\GoToFrame{0}{10}{$\sin \frac{\pi}{2}$}
\GoToFrame{0}{20}{$\sin \pi$}
\GoToFrame{0}{30}{$\sin \frac{3}{2}\pi$}
\GoToFrame{0}{40}{$\sin 2\pi$}}

\end{center}
\end{document} 
