\documentclass[english,russian]{article}
\usepackage[T2A]{fontenc}
\usepackage[utf8x]{inputenc}
\usepackage[russian]{babel}
\usepackage{graphicx}
\usepackage{fullpage}
\usepackage[x11names,dvipsnames,usenames]{xcolor}
\usepackage{tikz}

\usetikzlibrary{mindmap,trees,shadows,backgrounds,shapes,fadings}

\definecolor{light-blue}{rgb}{0.8,0.85,1}

\begin{document}
\pagestyle{empty}
\begin{flushleft}
\begin{tikzpicture}
  \path[mindmap,concept color=yellow, text=white,
  	every node/.style={concept, circular drop shadow,execute at begin
  		node=\hskip0pt, level distance=3cm, sibling angle=90}]
    node[concept, concept color=yellow, line width=1ex, text=white,
		 fill=black, font=\large\scshape, circular drop shadow,execute at
			begin node=\hskip0pt] (concept) {Материя}
    [clockwise from=-60]
    child[concept color=green!50!black] {
	node[concept, line  width=1ex, fill=magenta!10,
		text=black]
		{Не имеет\\ Структуру}
	[clockwise from=300]
	child { node[concept] {Энергия, ЭМ Поле}
	  [clockwise from=0]
	  child [concept color=black!50!magenta!90 ]{ node[concept] {\scalebox{0.9}{Излучение}}}
	 	 child [sibling angle=180, concept color=black!80!blue!90] { node[concept] {Скрытая\\ Энергия}}}
	  }
    child[concept color=blue] {
      node[concept, concept color=blue, line  width=1ex, fill=magenta!10,
	  	text=black]
	  	{Имеет\\ Структуру}
	  [clockwise from=-120]
	  child { node[concept] {Вещество}
	  [clockwise from=0]
	  child [concept color=black!80!blue!90] { node[concept] {Скрытое\\ вещество}}
	  child [sibling angle=180, concept color=blue!70!cyan!50, text=black] { node[concept] {Обычное\\ вещество}}}
    };
\end{tikzpicture}
\end{flushleft}
\end{document}
